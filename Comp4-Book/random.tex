\chapter{\code{<random>}}\label{ch:random}

Many things in life happen seemingly at random.
When we roll dice, any of the six faces (for standard cube dice) could come up with equal probability (for ideal "fair" dice).
For a well-shuffled deck of cards, we could draw any of the cards, but that card is not available for future draws.

Yet, computers are by and large not random.
If you put in the same input, you always get the same output.
These systems are \emph{deterministic}.
While stochastic measures that measure real phenomena to generate random numbers exist,
computer usually simulate randomness by using a special \emph{seed} value and transform it using some sort of algorithm to generate a series of pseudo-random numbers.
These numbers are not actually random, but they have a number of statistical properties that are close enough for most purposes.

There are several algorithms for generating pseudo-random numbers, each with their own pros and cons.
The C++ \code{<random>} library includes three: a linear congruential generator (LCG), a Mersenne twister, and subtract-with-carry (also known as subtract-with-borrow).
Additionally, there is a \code{random\_device} that uses stochastic processes to generate true random numbers.

\section{Linear Congruential Generator (LCG)}\label{sec:random-lcg}

An LCG is a pseudo-random number generator defined by a recurrence relation 
$$x_{i+1} = (a\cdot{x_i} + c) \% m$$
where $x_i$ is the previous random number (with $x_0$ being the seed), $x_{i+1}$ is the next random number, and $a$, $c$, and $m$ are constants which are particular to the chosen generator.
Not all choices for $a$, $c$, and $m$ give useful generators.
In particular, $m$ is tied to the period (how long the sequence goes before repeating) and so should be relatively large.

The \code{minstd\_rand} and \code{minstd\_rand0} classes are both LCG implementations.
Overall, the \code{minstd\_rand} class has better randomness with identical overhead and so should be used in preference to \code{minstd\_rand0}.
The chosen constants for \code{minstd\_rand} are $a = 48271$ (a prime number), $c = 0$, and $m = 2^{31} - 1$ (a Mersenne prime).

Overall, LCGs are very fast and memory efficient, requiring only a single numeric field for the seed (as the constants are often hardcoded) and a single multiplication, addition, and division operation.
This makes them good for situations where lots of random numbers must be generated quickly, when multiple independent sequences of random numbers are required, or when memory is tight.
The randomness ranges from passable (for good choices of constants) to awful (for bad choices) and so they are not suitable for cases where high-quality randomness is required (such as for cryptography).
Finally, the period length is relatively short (being at most $m$) and so is not suitable for long sequences.

\section{Mersenne Twister}\label{sec:random-mt}

A Mersenne twister is a pseudo-random number generators that maintains an array of $n$ $w$-bit numbers and uses a matrix recurrence relation to "twist" the numbers.
Note that the matrix is chosen such that the result can be easily computed with a series of bit shifts and bit masks rather than doing a full matrix multiplication.

The \code{mt19934} and \code{mt19934\_64} classes for generating 32-bit (\code{uint32}) and 64-bit (\code{uint64}) numbers, respectfully.

Mersenne twisters have incredibly long periods ($2^{19934} - 1$ for the most common version), making them especially good when a long sequence of numbers is required.
Unfortunately, they have a relatively long warm-up period where similar seed numbers will give correlated sequences before they diverge.
This makes them unsuitable for repeated experiments such as Monte Carlo simulations.
They also require a large amount of memory ($\sim\kb{2.5}$) making them unsuitable for multiple parallel streams of random numbers, and are relatively slow making them unsuitable for situations where numbers must be generated quickly.

\section{\code{random\_device}}\label{sec:random-device}

The LCG and Mersenne twister are both pseudo-random number generators;
they use a formula to generate new numbers from previous numbers rather than using a random process.
In contrast, the \code{random\_device} class uses a stochastic process (like measuring the least-significant bit of the system clock on a key press or measuring stellar radiation with specialized hardware) to generate random numbers in a manner not unlike flipping coins.
Note that the process used is system dependent and may not be available on all systems.

There are pros and cons to this.
Since PRNGs are reproducible, they are useful for simulations or video games where the results need to be repeatable (such as for saving replays or synchronizing between players).
Certain encryption techniques also essentially boil down to synchronizing PRNGs between the sender and receiver and using the generated numbers to modify the cypher.
On the flip side, many other security situations \emph{don't} want this property because an adversary may be able to pick a seed that reduces the amount of protection afforded.
Note that there are separate cryptographically secure pseudo-random number generators that should be used for security rather than either the LCG or Mersenne twister.

Hardware RNGs require making some sort of stochastic measurement, a process which is comparatively slow.
Generally, the operating system will measure entropy over time from some noise source and store the results in a file (\code{/dev/random} on Linux).
When randomness is required, bits are consumed from this file which will be slowly refilled over time.
This means that if a large number of random numbers are requested in a short period of time the request may block until further measurements can be taken.

\section{Seeding the Generator}\label{sec:random-seed}

PRNGs like the LCG and Mersenne twister are initialized with a \emph{seed} value.
Two copies of the same PRNG seeded with the same value will generate the same sequence.
This is useful for being able to rerun a simulation (allowing the exact same experiment to be rerun) or for sharing procedurally-generated content across multiple players (since each player can generate the same content from the same seed).

In most situations, you want this seed to be picked essentially at random.
One common strategy is to use the system clock since the least significant bits of a single timestamp are essentially random.
Care must be taken to only do this once though; subsequent calls will yield numbers that are highly correlated with the first (and may be identical if your system clock resolution is poor enough).

\begin{lstlisting}
unsigned int seed = std::chrono::system_clock::now().time_since_epoch().count();
std::minstd_rand gen(seed);
\end{lstlisting}

Alternately, you can use a \code{random\_device} to generate a single random number which is used to seed your PRNG.
While \code{random\_device} is slow and can block if excessive numbers are requested, generating a single number should be fine.

\begin{lstlisting}
std::random_device rd;
std::mt19937 gen(rd());
\end{lstlisting}

\section{Distributions}\label{sec:random-dist}

The \code{()} operator for each of the random number generator classes creates a single integer between the \code{min()} and \code{max()} defined for that specific generator.
Generally speaking, this will not be the desired range.
The \code{<random>} library defines several distribution classes which can be used to generate numbers in different ranges and with particular distributions.
The simplest of these are \code{uniform\_int\_distribution} and \code{uniform\_read\_distribution} which generate numbers uniformly at random in the specified range.
Both are templated so that they can generate different kinds of numbers and take two constructor parameters $a$ and $b$ for the lower and upper bounds of their ranges.
Note that the \code{uniform\_int\_distribution} produces the closed range $[a,b]$ and \code{uniform\_real\_distribution} produces the semi-open range $[a,b)$.

Other distribution objects will produce different distributions.
For example, the \code{normal\_distribution} will produce a normal distribution with a mean $\mu$ and a standard deviation of $\sigma$.

\begin{lstlisting}
unsigned int seed = std::chrono::system_clock::now().time_since_epoch().count();
std::minstd_rand gen(seed);
std::uniform_int_distribution<int> d6(1, 6); // Standard die
int roll = d6(gen) + d6(gen); // Sum of two dice
\end{lstlisting}

\section{Dice Example}\label{sec:random-dice}

Below we present a \code{Dice} class that can be used for simulating dice for a bunch of games.
There are two constructors.
The default constructor initializes the random-number generator using the system clock so that each instantiation will generate a different sequence of numbers.
The second constructor takes a seed that is used to initialize the random-number generator.
This allows the same sequence to be replayed.

This class uses the \code{minstd\_rand} generator, but the \code{mt19934} generator could have been used instead.
The generator is a member variable so that it generates a single sequence of numbers rather than restarting the sequence after each roll.

The \code{roll()} method takes parameters for the number and size of the dice, but have default values for the most common option of a single six-sided die.
It creates a \code{uniform\_int\_distribution} which it uses to generate numbers in the desired range before discarding it after the sum has been tallied.

\lstinputlisting{code/dice.hpp}
\lstinputlisting{code/dice.cpp}


