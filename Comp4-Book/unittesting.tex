\chapter{Unit Testing}\label{ch:UnitTesting}

\section{Motivation}\label{sec:testing-motivation}

When you write a program, it can be difficult to tell if it does what you want it to.
There can be any number of errors in the program, some of which may be hard to identify.
Furthermore, once you fix one problem, you have to ensure that you didn't inadvertently introduce a new bug in the process.
Running your program and testing everything by hand is a very time-consuming process.
It is also easy to miss something, especially when you are rechecking a section for the umpteenth time.

Automated unit testing has us writing test programs that will run portions of our main program to ensure that they behave as expected.
These test programs are made up of a suite of small ``unit tests'' that each run a small portion of the overall code and verifies that it performs as expected.
If the developer accidentally makes a change to the behavior of the system, these tests will identify the change in behavior.
By running these tests frequently, these mistakes can be caught quickly and fixed before too many other changes have also been made.
This makes debugging much easier.

There are a number of different unit testing frameworks, both for C++ and for other languages.
The Boost Unit Test Framework is one such framework.


\section{Simple Test}\label{sec:testing-simple}

Below is a simple test for the \code{std::string} class.

\begin{lstlisting}
#define BOOST_TEST_DYN_LINK
#define BOOST_TEST_MODULE StringTest
#include <string>
#include <boost/test/unit_test.hpp>

BOOST_AUTO_TEST_CASE(TestSize) {
	std::string str = "hello, world!";
	BOOST_REQUIRE(str.size() == 13);
}
\end{lstlisting}

Here, the \code{<string>} library is the component that we are testing and \code{<boost/test/unit\_test.cpp} is the header for the test library.
The \code{BOOST\_TEST\_DYN\_LINK} macro is an option that tells the library that it should create a \code{main} function that will run your tests.
Without it, you would have to write your own \code{main} function and call the tests yourself.
The \code{BOOST\_TEST\_MODULE} defines a name for this suite of tests (\code{StringTest} in this case).
Both of these macros must be defined \emph{before} including the test library.

\code{BOOST\_AUTO\_TEST\_CASE} is a macro that will create a test function (here named \code{TestSize}), including a lot of code to add it to the test suite that you don't have to worry about.
You can have as many \code{BOOST\_AUTO\_TEST\_CASE}s as you want;
usually you will have a lot of them that each test specific things.
Note that since they create functions, you need to obey all of the usual C++ naming rules including about having multiple functions with identical names.

Inside the test, we use the \code{BOOST\_REQUIRE} macro to create an assertion that the string has the correct size.
If the condition inside the macro is \code{false}, then the test ends in failure.
If it is \code{true}, then the test continues.
If the test ends normally, then it passes.

There is also a \code{BOOST\_CHECK} macro.
Like \code{BOOST\_REQUIRE}, \code{BOOST\_CHECK} will cause the test to fail if the condition is \code{false}.
Unlike \code{BOOST\_REQUIRE}, the test continues even on a failure.
This means that a single test could have multiple failures.
You use \code{BOOST\_CHECK} when testing multiple independent assertions such as functions without side-effects
and \code{BOOST\_REQUIRE} when an earlier failure implies that later assertions will also fail (such as when modifying an object).


\section{Comparisons}\label{sec:testing-compare}

The assertion in Chapter \ref{sec:testing-simple} has a limitation;
if the assertion fails, we don't know \emph{why} it failed (we don't know what the size is).
Instead, we can use the \code{BOOST\_REQUIRE\_EQUAL}.
If this fails, it will report what the two values are using the \oplshift operator.
This gives us more information about what went wrong.

There are a number of different assertion macros that we can use for different types of comparisons.
Each comparison has both a \code{REQUIRE} and a \code{CHECK} version, depending on whether we want to stop or continue on a failure.
A list of these macros can be seen in Table \ref{tbl:test-macro}

\begin{table}
	\begin{tabular}{|l|l|l|}
	\hline
	Operator& Require Macro & Check Macro \\ \hline
	== & \code{BOOST\_REQUIRE\_EQUAL} & \code{BOOST\_CHECK\_EQUAL} \\ \hline
	!= & \code{BOOST\_REQUIRE\_NE} & \code{BOOST\_CHECK\_NE} \\ \hline
	< & \code{BOOST\_REQUIRE\_LT} & \code{BOOST\_CHECK\_LT} \\ \hline
	<= & \code{BOOST\_REQUIRE\_LE} & \code{BOOST\_CHECK\_LE} \\ \hline
	> & \code{BOOST\_REQUIRE\_GT} & \code{BOOST\_CHECK\_GT} \\ \hline
	>= & \code{BOOST\_REQUIRE\_GE} & \code{BOOST\_CHECK\_GE} \\ \hline
	\end{tabular}
	\caption{Comparison Macros}\label{tbl:test-macro}
\end{table}

Since floating-point numbers (like \code{float} and \code{double}) can have a small amount of error, there also exists a \code{BOOST\_REQUIRE\_CLOSE} macro that can determine if two floating-point numbers are within a specified threshold of each other.

Here is a revised test case for the example in Section \ref{sec:testing-simple}.


\begin{lstlisting}
BOOST_AUTO_TEST_CASE(TestSize) {
	std::string str = "hello, world!";
	BOOST_REQUIRE_EQUAL(str.size(), 13);
}
\end{lstlisting}


\section{Exceptions}\label{sec:testing-except}

It is important to test both the success cases and failure cases.
For example, we should check that the program behaves properly when given values outside of the allowable range.
Often times, the correct behavior is to throw an exception.
We can use the \code{BOOST\_REQUIRE\_THROW} or \code{BOOST\_CHECK\_THROW} macros to verify that a piece of code throws a particular kind of exception.
The second argument is the type of exception that is expected, so if it throws the wrong kind of exception it will fail the test.
Note that if a subclass of the exception is thrown it will still pass the test.
Conversely, if you want to check that it \emph{doesn't} throw an exception, you can use \code{BOOST\_REQUIRE\_NO\_THROW} or \code{BOOST\_CHECK\_NO\_THROW}.

\begin{lstlisting}
BOOST_AUTO_TEST_CASE(TestAt) {
	std::vector<int> v = {1};
	BOOST_REQUIRE(v.empty());
	BOOST_REQUIRE_NO_THROW(v.at(0));
	BOOST_REQUIRE_THROW(v.at(1), std::out_of_range);
}
\end{lstlisting}


\section{Test-Driven Development}\label{sec:testing-tdd}

When we write a new test and it passes it raises a question; did it pass because the code it is testing is good, or because the test is not good enough to detect problems?
For example, we could write a trivial test that does nothing and always passes.
This test would be useless for detecting problems.

The key to answering the question is to write the test first, \emph{before} the code which it tests.
Since the feature being tested doesn't exist yet, the test will fail.
Once we implement the feature, the test should pass.
In this way, the test becomes a representation of what it means to have implemented the feature.

When we write a new feature, we follow these steps.

\begin{enumerate}
	\item Write some tests that define the expected behavior for the the new feature.  These tests may not compile if they call new functions.
	\item Write a stub for these new functions.
	\item Run the tests.  The new tests should fail since the feature hasn't been implemented yet.
	\item Write the new feature.
	\item Rerun the tests.  If the feature was implemented properly, all of the tests will pass.  Fix any problems until this is the case.
	\item Refactor the code to make it more readable.  Rerun the tests to make sure you didn't accidentally break anything.
\end{enumerate}

