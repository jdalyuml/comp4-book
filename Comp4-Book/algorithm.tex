% !TeX root=Comp4-Book
\chapter{\code{<algorithm>}}\label{ch:algorithm}
\section{Motivation}\label{sec:algo-motivation}


\section{\code{accumulate}}\label{sec:algo-accumulate}

The \code{std::accumulate} represents a \href{https://en.wikipedia.org/wiki/Fold_(higher-order_function)}{fold}
where each element in the sequence is repeatedly "`folded"' into a result using a provided combining function and initial result.
Unlike the other functions included in this chapter, \code{accumulate} is located in the \code{<numeric>} library.
The default version uses addition as the combining function to calculate the sum of the elements.
For example, the following code segment computes the sum of a \vectortype.

\begin{lstlisting}
vector<int> v = {1, 2, 3, 4, 5};
int sum = accumulate(v.begin(), v.end(), 0); // 15
\end{lstlisting}

There is also an overload that takes the combining function as an argument.
The following code segment uses the \code{multiplies} class from the \code{<function>} library to compute the product of the elements.
Note that 1 is used for the initial result since that is the multiplicative identity.

\begin{lstlisting}
vector<int> v = {1, 2, 3, 4, 5};
int product = accumulate(v.begin(), v.end(), 1, multiplies<int>()); // 120
\end{lstlisting}

Since the \code{accumulate} function is generic, you can use it with any type that has a $+$ operator.
This code segment will concatenate several strings together into a single string.

\begin{lstlisting}
vector<string> v = {"hello", "world"}
string concat = accumulate(v.begin(), v.end(), string()); // "helloworld"
\end{lstlisting}


\section{\code{find}}\label{sec:algo-find}
find, findif, and search
\section{\code{any\_of}}\label{sec:algo-any}
anyof, allof, and noneof
\section{\code{count}}\label{sec:algo-count}
count and countif
\section{min and max}\label{sec:algo-minmax}
\section{equal}\label{sec:algo-equal}
\section{copy}\label{sec:algo-copy}
copy, copyif
\section{transform}\label{sec:algo-transform}
\section{move}\label{sec:algo-move}
\section{fill and generate}\label{sec:algo-fill}
\section{replace}\label{sec:algo-replace}
\section{sort}\label{sec:algo-sort}
sort, stablesort, partialsort
binarysearch, includes, lowerbound, upperbound


