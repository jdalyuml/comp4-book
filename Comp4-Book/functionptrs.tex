\chapter{Function Pointers}\label{ch:FuncPtrs}

\section{Motivation}\label{sec:fp-motivation}

There are times where we want to have variations on a piece of code where we inject a single behavioral change.
For example, consider the following code that insertion sorts a vector of strings.

\begin{lstlisting}
void insertionSort(std::vector<std::string>& v) {
	for (size_t i = 1; i < v.size(); i++) {
		std::string s = v[i];
		size_t j;
		for (j = i; j > 0 && v[j - 1] < s; j--) {
			v[j] = v[j - 1];
		}
		v[j - 1] = s;
	}
}
\end{lstlisting}

What if we wanted to compare the strings in a case-insensitive manner?
We would write a very similar function.

\begin{lstlisting}
bool ltIgnoreCase(const std::string& s1, const std::string& s2) {
	for (size_t i = 0; i < s1.size() && i < s2.size(); i++) {
		if (std::tolower(s1[i]) != std::tolower(s2[i])) {
			return std::tolower(s1[i]) < std::tolower(s2[i]);
		}
	}
	return s1.size() < s2.size();
}
void insertionSortIgnoreCase(std::vector<std::string>& v) {
	for (size_t i = 1; i < v.size(); i++) {
		std::string s = v[i];
		size_t j;
		for (j = i; j > 0 && ltIgnoreCase(v[j - 1], s); j--) {
			v[j] = v[j - 1];
		}
		v[j - 1] = s;
	}
}
\end{lstlisting}

In this example, the \code{insertionSortIgnoreCase} function is the same as the \code{insertionSort} function except that it uses the \code{ltIgnoreCase} function in place of the \code{<} operator.
We can imagine a number of variants to this that use a variety of alternate comparisons here (e.g. using a greater than comparison to reverse sort),
especially if we templated the function so that it could work for any kind of \code{vector}.

\section{Function Pointers}

The code for a function exists in memory, which means that it has an address.
A pointer stores an address, which means that a pointer variable can store the address of a function.
For a comparison function like the \code{ltIgnoreCase} function above, we could write something like:
\begin{lstlisting}
bool (*comp)(const std::string&, const std::string&) = &ltIgnoreCase;
\end{lstlisting}
We read this as \code{comp} is a pointer to a function that takes two \code{string}s and returns a \code{bool}, and is bound to \code{ltIgnoreCase}.
Note that there are no parenthesis on \code{ltIgnoreCase};
we are getting the address of the function itself, not for whatever invoking the function would return.
Also, the \code{\&} is optional, so this would also have been acceptable:
\begin{lstlisting}
bool (*comp)(const std::string&, const std::string&) = &ltIgnoreCase;
\end{lstlisting}

At this point, we can treat \code{comp} just like a function and invoke it as though it were the function it points to:
\begin{lstlisting}
bool result = comp("hello", "world"); //  evaluates to true
\end{lstlisting}

Putting it all together (with some templates to make it more useful), we get the following function:

\begin{lstlisting}
template <typename T>
void insertionSort(std::vector<T>& v, bool (*comp)(const T&, const T&)) {
	for (size_t i = 1; i < v.size(); i++) {
		T s = v[i];
		size_t j;
		for (j = i; j > 0 && comp(v[j - 1], s); j--) {
			v[j] = v[j - 1];
		}
		v[j - 1] = s;
	}
}
\end{lstlisting}


\section{\code{Function}}

The \code{<functional>} library defines a lot of tools for working with functions.
One of these tools is a class \code{function} that wraps a function pointer or function-like object (one that defines an \code{operator()}).
This allows us to create variables that can represent not just functions, but also to things that act like functions.
Additionally, the notation is a little easier to understand; the previous notation may not immediately be obvious (especially to newer users) that it represents a function.

\begin{lstlisting}
std::function<bool(const std::string&, const std::string&)> comp = &ltIgnoreCase
\end{lstlisting}

The \code{<functional>} library also defines a number of objects that represent function calls, such as \code{plus} (representing the \code{+} operator) and \code{equal\_to} (representing the \code{==}) operator).
A list of these function objects can be seen in Table \ref{tbl:FunctionObjects}.

\begin{table}
	\centering
	\begin{tabular}{|l|l|l|}
		\hline
		Arithmetic & Relational & Logical \\ \hline
		\hline
		\code{plus<T>} & \code{equal\_to<T>} & \code{logical\_and<T>} \\ \hline
		\code{minus<T>} & \code{not\_equal\_to<T>} & \code{logical\_or<T>} \\ \hline
		\code{multiplies<T>} & \code{greater<T>} & \code{logical\_not<T>} \\ \hline
		\code{divides<T>} & \code{greater\_equal<T>} & \\ \hline
		\code{modulus<T>} & \code{less<T>} & \\ \hline
		\code{negate<T>} & \code{less\_equal<T>} & \\ \hline
	\end{tabular}
	\caption{Function objects representing operators}\label{tbl:FunctionObjects}
\end{table}

Putting this all together, we create the final version of our \code{insertionSort} function that uses the \code{less} function object to sort items according to their natural ordering if no comparison function is provided.

\begin{lstlisting}
template <typename T>
void insertionSort(
		std::vector<T>& v, 
		std::function<bool(const T&, const T&)> comp = std::less<T>()) {
	for (size_t i = 1; i < v.size(); i++) {
		T s = v[i];
		size_t j;
		for (j = i; j > 0 && comp(v[j - 1], s); j--) {
			v[j] = v[j - 1];
		}
		v[j - 1] = s;
	}
}
\end{lstlisting}


\section{Lambda Expression}

The version of our \code{insertionSort} function makes sorting in non-standard orders significantly more convenient.
As long as we have an appropriate comparison function, we can order the items according to that function.
This does mean that for each different ordering we might want to use will require a separate one of these helper functions.
Many of these functions will be used in only one place and our namespace may become littered with a bunch of these little functions that are used in only one place.

Since C++11, we can alleviate this by using \emph{lambda expressions}.
Lambda expressions are statements that create \emph{closures}, which bind some local variables into an anonymous function.
Generally, a lambda expression in C++ will take the form of:
\begin{lstlisting}
[capture] (params) -> type { body }
\end{lstlisting}

Here, \code{params} is the normal parameter list of the function, \code{type} is the return type of the function, and \code{body} is the usual function body.
Note that the return type comes \emph{after} the parameter list rather than before like with regular functions, and can usually be intuited and is optional in most cases.
Likewise, the \code{auto} keyword is often used for the parameter variable types when their types can be intuited from the context.

The capture list is new.
This is a comma separated list or any local variables that are needed in the closure.
The closure has access to any variables that are captured by the lambda expression.
These variables can either by captured by value using \code{=}, in which case their value is copied into the closure,
or captured by reference using \code{\&}, in which case references to the variable are given to the closure.
Similar to regular functions, any changes made to a variable captured by value will not be reflected in the surrounding context, by changes to ones captured by reference will be.
If you need to capture multiple variables, either you can list each one or you can place either just an \code{=} or an \code{\&} to capture all variables.

The capture list is required, even if no variables are captured.
This will mean that the closure will not have access to any variables from the surrounding environment.
For example, here is a call to our \code{insertionSort} function with a lambda expression to sort in reverse order.
\begin{lstlisting}
insertionSort(words, [](auto a, auto b) { return a > b; });
\end{lstlisting}
In this case, we only need to compare the two parameters \code{a} and \code{b}, so we don't need any variables from the local scope.
Likewise, the types of the variables can be intuited from context and so the \code{auto} keyword can be used.
Note that in this case, we could have used the \code{std::greater} function object to achieve the same effect.
